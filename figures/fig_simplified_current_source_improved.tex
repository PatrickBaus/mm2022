\documentclass[tikz]{standalone}

\usepackage{tikz}
\usepackage{siunitx}
\usepackage{circuitikz}

\begin{document}
    \usetikzlibrary{arrows, fit}
    \begin{circuitikz}[
        scale=0.45,
        transform shape,
        european resistors,
    ]
        % Help lines for debugging
        %\draw[help lines] (0,-8) grid (30,0);
        \tikzstyle{every node}=[font=\Large]
        % FET currenct source
        \draw
        (0,0) node[left, align=right](Vdiode) {V$_{\text{diode}}$\\\qty{12}{\V}} to [short, o-] (3,0)
        to [short, *-] ++(0,-3)
        ++(0,-1) node[draw,thick,minimum width=2cm,minimum height=2cm](DAC){DAC}
        ++(4,0.5) node[op amp](opamp){}
        (opamp.+) -- (DAC.east)
        (DAC.south) to [short, -o] ++(0,-0.5) node[below] {V$_{\text{ref}}$}
        (DAC.west) to [short, -o] ++(-0.5,0) node[left] {to MCU}
        (opamp.out) --++ (1,0) node[pigfete, anchor=gate](FET){}
        (FET.S) --++ (0,1) coordinate (Rs) to [R, l^=R$_{\text s}$, *-] ++(0,2) -- (2,0)
        (opamp.-) -|++ (-0.5,1) -- (Rs)
        (FET.D) --++ (0,-0.5) to [led, -] ++(0,-1.5)
        ($(FET.gate)+ (0,-3.5)$) node[pigfete, anchor=gate](FET2){}
        ($(opamp) + (0,-3.5)$) node[op amp](opamp2){}
        (opamp2.out) -- (FET2.G)
        (opamp2.-) --++ (-0.5,0) |- ($(FET.D) + (0,-0.5)$) coordinate(virtualGnd)
        (opamp2.+) --++ (-0.5,0) node[ground]() {}
        (FET2.D) node[vee]{$-\text V_{\text{diode}}$} node[below](Vneg){}
        ;
        \node[draw=red, rounded corners=2pt, fit={(Vdiode) ($(Vneg) +(0,-1)$) ($(FET.east) + (0.75,0)$)}] (ConstantCurrentSource){};
        \node[red, above] at (ConstantCurrentSource.north) {Constant Current Source};
        % Howland current source
        \draw
        (Rs) ++ (4,0) node[op amp, xscale=-1](opamp2){}
        ($(Rs) + (2,0)$) coordinate(Opamp2Output)
        (opamp2.out) -- (Opamp2Output) |-++ (1,2) to [R, -*] ++(3,0) coordinate(R1)
        to [R, -] ++ (3,0) node[ground]() {} node[anchor=west](ModInInv){}
        (R1) |- (opamp2.-)
        (Opamp2Output) to [short, *-] ++(0,-2)
        --++(1,0) to [R, -*] ++(3,0) coordinate(R3)
        to [R, -] ++ (3,0) node[ocirc] {} node[anchor=west](ModIn) {Mod In}
        (R3) |- (opamp2.+)
        ($(FET.D) + (0,-0.5)$) to [short, *-] ++(2,0) coordinate(Rb) to [R, -] ++(2,0) -| (R3)
        ;
        \node[draw=blue, rounded corners=2pt, fit={(ModIn) ($(Rb) + (0,-0.25)$) ($(Opamp2Output) + (-0.25,0)$) ($(ModInInv) + (0,0.25)$)}] (HowlandCurrentSource){};
        \node[blue, above] at (HowlandCurrentSource.north) {Modulation Current Source};
        % Annotations
        % Virtual ground label
        \node[orange, font=\bfseries](vGndLabel) at (14,-7) {\LARGE Virtual Ground};
        \path[orange, ->, very thick, >=stealth'] (vGndLabel) edge node {} (virtualGnd);
        % Current source
        \node[draw=orange, very thick, rounded corners=2pt, fit={(opamp) ($(Rs) + (0.25,2)$)}] (CurrentSource){};
        \node[orange, font=\bfseries](2xCurrentSource) at (12,1) {\LARGE 2x};
        \path[orange, ->, very thick, >=stealth'] (2xCurrentSource) edge node {} (CurrentSource);
    \end{circuitikz}
\end{document}
