\documentclass[tikz,crop]{standalone}

\usepackage{tikz}
\usepackage{siunitx}
\usepackage{circuitikz}


\begin{document}

    \usetikzlibrary{fit,arrows.meta,positioning}
    \begin{circuitikz}[
        scale=0.45,
        transform shape,
        european resistors,
    ]
            % Help lines for debugging
            %\draw[help lines] (0,-6) grid (30,0);
            \tikzstyle{every node}=[font=\Large]
            % FET currenct source
            \draw
            (0,0) node[left, align=right](Vdiode) {V$_{\text{diode}}$\\\textcolor{green!50!black}{\qty{12}{\V}}} to [short, o-] (2,0)
            to [zD, l_=LM399, i>_=\qty{1}{\mA}, invert, *-*, name=lm399] ++(0,-5)
            to [R, -, l_=R$_{\text z}$] ++(0,-2)  --++ (0,-0.5) node[vee]{\qty{-15}{\V}} node[below](Vneg){}
            (2,0) -|++ (2,-3) to [pR, n=POT, l_=R$_{\text{set}}$] ++(0,-2) --+ (-2,0)
            (POT.wiper) ++(2.5,0.5) node[op amp](opamp){}
            (opamp.+) -- (POT.wiper)
            (opamp.out) --++ (1,0) node[pigfete, anchor=gate](FET){}
            (FET.S) --++ (0,1) coordinate (Rs) to [R, l^=R$_{\text s}$, *-, name=Rsense] ++(0,2) to [short, -*] (4,0)
            (opamp.-) -|++ (-0.5,1) -- (Rs) (Rs) -- (FET.S)
            (opamp.out) -- (FET.G)
            (FET.D) --++ (0,-0.5) to [led, name=laser diode] ++(0,-2)
            node[ground]() {}
            ;
            \node[draw=red, rounded corners=2pt, fit={(Vdiode) ($(Vneg) +(0,-1)$) ($(FET.east) + (0.5,0)$)}] (ConstantCurrentSource){};
            \node[red, above] at (ConstantCurrentSource.north) {Unidirectional Constant Current Source};
            % Howland current source
            \draw
            (Rs) ++ (4,0) node[op amp, xscale=-1](opamp2){}
            ($(Rs) + (2,0)$) coordinate(Opamp2Output)
            (opamp2.out) -- (Opamp2Output) |-++ (1,2) to [R, -*] ++(3,0) coordinate(R1)
            to [R, -] ++ (3,0) node[ground]() {} node[anchor=west](ModInInv){}
            (R1) |- (opamp2.-)
            (Opamp2Output) to [short, *-] ++(0,-2)
            --++(1,0) to [R, -*] ++(3,0) coordinate(R3)
            to [R, -] ++ (3,0) node[ocirc] {} node[anchor=west](ModIn) {Mod In}
            (R3) |- (opamp2.+)
            (R3) |- ($(FET.D) + (0,-0.5)$) node[circ]{}
            ;
            \node[draw=blue, rounded corners=2pt, fit={(ModIn) ($(Opamp2Output) + (-0.25,0)$) ($(ModInInv) + (0,0.25)$)}] (HowlandCurrentSource){};
            \node[blue, above] at (HowlandCurrentSource.north) {Bidirectional Modulation Current Source};
            % Annotations
            \draw node[circle,  draw=green!50!black,  minimum size=35, thick] (lm399 circ) at (lm399) {};
            \draw node[circle,  draw=green!50!black,  minimum size=43, thick] (rs circ) at (Rsense) {};
            \draw node[circle,  draw=green!50!black,  minimum size=35, thick] (laser circ) at (laser diode) {};
            \draw node[circle,  draw=green!50!black,  minimum size=35, thick] (fet circ) at (FET.circle center) {};

            \draw node[above=1.5,  green!50!black] at (opamp.north) (annotation) {\qty{7}{\V}};
            \draw node[green!50!black, below left=of laser diode] (annotation ld) {\qtyrange{2.1}{7}{\V}};
            \draw node[green!50!black, below left=of FET] (annotation fet) {?};

            \draw[-Stealth,  draw=green!50!black] (annotation) -- (lm399 circ);
            \draw[-Stealth,  draw=green!50!black] (annotation ld) -- (laser circ);
            \draw[-Stealth,  green!50!black] (annotation) -- (rs circ);
            \draw[-Stealth,  green!50!black] (annotation fet) -- (fet circ);
    \end{circuitikz}
\end{document}
